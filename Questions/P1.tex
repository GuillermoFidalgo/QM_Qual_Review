
\section[Wave Functions]{Wave Functions; State Functions and operators; uncertainty principle}
\subsection{Wave Functions}

\begin{frame}{Wave Functions}
The Schrödinger equation describes how the quantum state of a physical system changes over time. The wave function, denoted by $\Psi$, is a complex-valued function that contains all the information about the system.

\begin{block}{Schrödinger Equation}
    \begin{equation*}
        i\hbar \frac{\partial \Psi(\mathbf{r}, t)}{\partial t} = \hat{H} \Psi(\mathbf{r}, t)
    \end{equation*}
    where $\hat{H}$ is the Hamiltonian operator, which represents the total energy of the system.

    In one dimension, the time-dependent Schrödinger equation can be written as:
    \begin{equation*}
        % -\frac{\hbar^2}{2m} \frac{d^2 \Psi(x)}{dx^2} + V(x) \Psi(x) = E \Psi(x)
        i\hbar \frac{\partial \Psi(x, t)}{\partial t} = -\frac{\hbar^2}{2m} \frac{d^2 \Psi(x, t)}{dx^2} + V(x) \Psi(x, t)
    \end{equation*}
    where $V(x)$ is the potential energy and $m$ is the mass of the particle.

\end{block}

\end{frame}

\begin{frame}{Wave Functions (cont)}
The wave function collapses when a measurement (i.e. observation) is made.
In QM, we may only measure the probability of observing the position of a particle at specific time $t$ by computing $|\Psi(x,t)|^2$ or more precisely
\begin{block}

    \[
        \int_a^b |\Psi(x,t)|^2 \dd{x} = \int_a^b \Psi^*(x,t) \Psi(x,t) \dd{x}
    \]

\end{block}
\end{frame}

\subsection{Stats review}

\begin{frame}
    \frametitle{Some important stats review}
\begin{description}
    \item[mean] is the numerical average of multiple measurements at a time.
    \item[median] the 50th percentile, second quantile, i.e. the value at which there
    is the same probability to measure any value below or after this value.
    \item[mpv] the number that has the highest probability to be measured.
    Another name for the ``mode''.
    \item[expectation value]  this a little misnomer. Same as the mean value in our case.
\end{description}
We comput these as follows
% this is a new change

\begin{onlyenv}<1>

    \begin{block}{mean/expectation value}
        The average is computed for a number of different moments of j as follows
        \begin{gather}
            \ev{j}= \sum_{j=0}^\infty j P(j) \qq{;} \ev{j^2} = \sum_{j=0}^\infty j^2 P(j) \qq{\dots} \ev{j^n} = \sum_{j=0}^\infty j^n P(j)\\
           \text{In General} \hspace{1cm} \boxed{\ev{f(j)} = \sum_{j=0}^\infty f(j) P(j)}
        \end{gather}

    \end{block}

\end{onlyenv}

\begin{onlyenv}<2>

    \begin{block}{median}
        The formula depends on the number of observations or data points. First we order the data list
        $\{X_1, X_2, \dots, X_n\}$ then
        \[
        \text{Median} =
        \begin{cases}
            \frac{X_{n/2} + X_{(n/2)+1}}{2} & \text{if } n \text{ is even} \\
            X_{(n+1)/2} & \text{if } n \text{ is odd}
        \end{cases}
        \]

    \end{block}

\end{onlyenv}

\end{frame}


\begin{frame}{Variance and Standard Deviation}
    These measure the spread of a distribution (particularly the standard deviation).
    First we find the deviation of each value from the mean.
    \[
        \Delta j = j - \ev j
    \]

    then we find the average of the \textit{square} of the deviations (why? Because the average deviation is \textbf{always} zero!). So now
    \begin{align*}
        \sigma^2 & =\ev{(\Delta j)^2} = \sum(\Delta j)^2 P(j)=\sum(j-\ev{j})^2 P(j) \\
        & =\sum\left(j^2-2 j \ev{j} + \ev{j}^2\right) P(j) \\
        & =\sum j^2 P(j)-2 \ev{j} \sum j P(j)+\ev{j}^2 \sum P(j) \\
        & =\ev{ j^2}-2\ev{ j}\ev{ j}+\ev{ j}^2=\ev{ j^2}-\ev{ j}^2 .
    \end{align*}

    Finally $$\sigma = \sqrt{\ev{ j^2}-\ev{ j}^2}$$
\end{frame}


\begin{frame}
    \frametitle{Probability density and properties for continuous functions}
    $$ P_{ab} = \int_a^b \rho(x) \dd{x}$$
    is the probability that $x$ lies between $a$ and $b$. The other properties are:
    \begin{gather}
    \int_{-\infty}^{+\infty} \rho(x) d x=1 \\
    \langle x\rangle=\int_{-\infty}^{+\infty} x \rho(x) d x \\
    \langle f(x)\rangle=\int_{-\infty}^{+\infty} f(x) \rho(x) d x \\
    \sigma^2 \equiv\left\langle(\Delta x)^2\right\rangle=\left\langle x^2\right\rangle-\langle x\rangle^2
\end{gather}

\end{frame}

\subsection{Back to the wave function}


% \subsection{State Functions and operators}
% \begin{frame}{State Functions and operators}

% \end{frame}



% \subsection{Uncertainty Principle}
% \begin{frame}{Uncertainty Principle}

% \end{frame}

% \begin{frame}{Uncertainty Principle}
% 	\begin{block}{Heisenberg's Uncertainty Principle}
% 		\begin{equation*}
% 			\Delta x \Delta p \geq \frac{\hbar}{2}
% 		\end{equation*}
% 		where $\Delta x$ is the uncertainty in position and $\Delta p$ is the uncertainty in momentum.
% 	\end{block}
% \end{frame}
