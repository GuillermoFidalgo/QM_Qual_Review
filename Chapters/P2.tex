\section{Schrödinger Equation}

\subsection[TISE]{Time Independent Schrödinger Equation}

\begin{frame}{Time Independent Schrödinger Equation}
	The time-independent Schrödinger equation is  given by:
	\begin{equation*}
		\hat{H} \psi(\mathbf{r}) = E \psi(\mathbf{r})
	\end{equation*}
	Another way to write it is:
	\begin{equation*}
		-\frac{\hbar^2}{2m} \nabla^2 \psi(\mathbf{r}) + V(\mathbf{r}) \psi(\mathbf{r}) = E \psi(\mathbf{r})
	\end{equation*}

\end{frame}







\subsection[TDSE]{Time Dependent Schrödinger Equation}

\begin{frame}{Time Dependent Schrödinger Equation}
	The time-dependent Schrödinger equation describes how the quantum state of a system evolves over time. It is given by:
	\begin{equation*}
		i\hbar \frac{\partial \Psi(\mathbf{r}, t)}{\partial t} = \hat{H} \Psi(\mathbf{r}, t)
	\end{equation*}
	where $\Psi(\mathbf{r}, t)$ is the wave function of the system, $i$ is the imaginary unit, $\hbar$ is the reduced Planck's constant, and $\hat{H}$ is the Hamiltonian operator representing the total energy of the system.
	\begin{block}{Hamiltonian Operator}
		The Hamiltonian operator $\hat{H}$ can be expressed as:
		\begin{equation*}
			\hat{H} = -\frac{\hbar^2}{2m} \nabla^2 + V(\mathbf{r})
		\end{equation*}
		where $m$ is the mass of the particle, $\nabla^2$ is the Laplacian operator, and $V(\mathbf{r})$ is the potential energy function.
	\end{block}
\end{frame}


\begin{frame}{Time Dependent Schrödinger Equation (cont)}
	The wave function $\Psi(\mathbf{r}, t)$ contains all the information about the quantum state of the system. The probability density of finding a particle at position $\mathbf{r}$ at time $t$ is given by:
	\begin{equation*}
		P(\mathbf{r}, t) = |\Psi(\mathbf{r}, t)|^2
	\end{equation*}
	The wave function must satisfy the normalization condition:
	\begin{equation*}
		\int |\Psi(\mathbf{r}, t)|^2 \, d^3r = 1
	\end{equation*}
\end{frame}

\begin{frame}
	\frametitle{Recipe to solve the general problem}

	\begin{itemize}
		\item You are given a time independent potential $V(x)$ and the starting wave function $\Psi (x,0)$. The goal is to find $\Psi(x,t)$.
		\item First solve the time independent Schrödinger equation. This in general, yields an infinite set of solutions $\{\psi_n(x)\}$ each with it's own energy $\{E_n\}$.
		\item Write down the general linear combinations of these solutions  $$ \Psi(x,0) = \sum_{n=1}^\infty c_n \psi_n(x)$$
		\item now add each of the corresponding time dependence factors $\exp(-i E_n t /\hbar)$
		      \[
			      \Psi(x,t) = \sum_{n=1}^{\infty} c_n \psi_n(x) e^{-iE_n t/ \hbar} = \sum_{n=1}^{\infty} c_n \Psi_n(x,t)
		      \]
		      Where $|c_n|^2$ is the probability that a measurement of the energy would return the value $E_n$.
	\end{itemize}
\end{frame}
